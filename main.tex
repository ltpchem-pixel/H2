%%%%%%%%%%%%%%%%%%%%%%%%%%%%%%%%%%%%%%%%%
% Journal Article
% LaTeX Template
% Version 2.0 (February 7, 2023)
%
% This template originates from:
% https://www.LaTeXTemplates.com
%
% Author:
% Vel (vel@latextemplates.com)
%
% License:
% CC BY-NC-SA 4.0 (https://creativecommons.org/licenses/by-nc-sa/4.0/)
%
% NOTE: The bibliography needs to be compiled using the biber engine.
%
%%%%%%%%%%%%%%%%%%%%%%%%%%%%%%%%%%%%%%%%%

%----------------------------------------------------------------------------------------
%	PACKAGES AND OTHER DOCUMENT CONFIGURATIONS
%----------------------------------------------------------------------------------------

\documentclass[
	a4paper, % Paper size, use either a4paper or letterpaper
	10pt, % Default font size, can also use 11pt or 12pt, although this is not recommended
	unnumberedsections, % Comment to enable section numbering
	twoside, % Two side traditional mode where headers and footers change between odd and even pages, comment this option to make them fixed
]{LTJournalArticle}

\addbibresource{sample.bib} % BibLaTeX bibliography file

\runninghead{Graphene oxide axial condensation} % A shortened article title to appear in the running head, leave this command empty for no running head

\footertext{\textit{Journal name} (2024) 12:533-684} % Text to appear in the footer, leave this command empty for no footer text

\setcounter{page}{1} % The page number of the first page, set this to a higher number if the article is to be part of an issue or larger work
\usepackage{mathtools}

%----------------------------------------------------------------------------------------
%	TITLE SECTION
%----------------------------------------------------------------------------------------

\title{Synthesis, crystal structure and vibrational spectra of the new strong and eco-friendly reducing agent
$\textrm{N}_{2}\textrm{H}_{6}(\textrm{H}_{2}\textrm{PO}_{2})_{2}$} % Article title, use manual lines breaks (\\) to beautify the layout

% Authors are listed in a comma-separated list with superscript numbers indicating affiliations
% \thanks{} is used for any text that should be placed in a footnote on the first page, such as the corresponding author's email, journal acceptance dates, a copyright/license notice, keywords, etc
\author{%
	 Ivo I. Lozanov\textsuperscript{1}\thanks{Corresponding author: \href{mailto:ivoil@uni-sofia.bg}{ivoil@uni-sofia.bg}\\},\,\,
     B. Morgenstern\textsuperscript{2},\,\,
     M. Tsvetkov\textsuperscript{1}\,\,
    and Lyudmil Lyutov\textsuperscript{1,3}
}

% Affiliations are output in the \date{} command
\date{\footnotesize\textsuperscript{\textbf{1}}Department of Inorganic Chemistry, Faculty of Chemistry and Pharmacy, Sofia University 'St. Kliment Ohridski'\\
\textsuperscript{\textbf{2}} Inorganic Solid State Chemistry, Saarland University, Campus Geb. C4 1, 66123 Saarbrücken, Germany; bernd.morgenstern@uni-saarland.de\\
\textsuperscript{\textbf{3}} Institute of Metal Science with Hydro- and Aerodynamics Centre 'Acad. Angel Balevski'}

% Full-width abstract
\renewcommand{\maketitlehookd}{%
	\begin{abstract}
		\noindent j
	\end{abstract}
}

%----------------------------------------------------------------------------------------

\begin{document}

\maketitle % Output the title section

\section{Introduction}


\section{Materials and Methods}

\subsection{Materials}
$\textrm{BaCl}_{2}\cdot2\textrm{H}_{2}\textrm{O}$ ($\geq$ 98$\%$,
Sigma-Aldrich, Milwaukee, WI, USA); $\textrm{NaH}_{2}\textrm{PO}_{2}\cdot\textrm{H}_{2}\textrm{O}$ ($\geq$ 99$\%$,
Sigma-Aldrich, Milwaukee, WI, USA);
$\textrm{H}_{2}\textrm{SO}_{4}$; $\textrm{N}_{2}\textrm{H}_{4}\cdot\textrm{H}_{2}\textrm{O}$ (64-65$\%$, $\geq$ 97$\%$, reagent grade
Sigma-Aldrich, Milwaukee, WI, USA). All chemicals were used
as purchased without further purification.

\subsection{Synthesis}

Hydrazinium bishypophosphite $\textrm{N}_{2}\textrm{H}_{6}(\textrm{H}_{2}\textrm{PO}_{2})_{2}$ was synthesized in two steps. Firstly a fresh
aqueous solution of hypophosphorous acid is prepared from Ba(H$_{2}$PO$_{2}$)$_{2}$ and H$_{2}$SO$_{4}$. Then the acid is neutralized
with hydrazine hydrate.

\subsection{Preparation of the precursor $\textrm{Ba}(\textrm{H}_{2}\textrm{PO}_{2})_{2}$}

Initially barium hypophosphite $\textrm{Ba}(\textrm{H}_{2}\textrm{PO}_{2})_{2}$ wa synthesized by
mixing freshly prepared saturated aqueous solutions of $\textrm{BaCl}_{2}$ and $\textrm{NaH}_{2}\textrm{PO}_{2}$ at
90 $^{\circ}\textrm{C}$. After cooling to $\sim 5\,^{\circ}\textrm{C}$ the reaction mixture was left for 2-3 h. Under
these conditions $\textrm{Ba}(\textrm{H}_{2}\textrm{PO}_{2})_{2}$ precipitates as a white crystalline product with
high yield. The obtained white crystals were recrystallized from water, washed with cool water and dried in a desiccator.\\

\begin{equation}
\textrm{BaCl}_{2} + 2\textrm{NaH}_{2}\textrm{PO}_{2} \rightarrow \textrm{Ba}(\textrm{H}_{2}\textrm{PO}_{2})_{2} \downarrow + 2\textrm{NaCl}.
\end{equation}

\subsection{Preparation of $\textrm{H}_{3}\textrm{PO}_{2}$ and $\textrm{N}_{2}\textrm{H}_{6}(\textrm{H}_{2}\textrm{PO}_{2})_{2}$}

Aqueous solution of hypophosphorous acid $\textrm{H}_{3}\textrm{PO}_{2}$ was prepared by dissolving
barium hypophosphite $\textrm{Ba}(\textrm{H}_{2}\textrm{PO}_{2})_{2}$ in cool water and slowly
adding stoichiometric amount of $\textrm{H}_{2}\textrm{SO}_{4}$ under constant stirring. A white crystalline
product of $\textrm{BaSO}_{4}$ precipitated immediately.

\begin{equation}
\textrm{Ba}(\textrm{H}_{2}\textrm{PO}_{2})_{2} + \textrm{H}_{2}\textrm{SO}_{4} \rightarrow 2\textrm{H}_{3}\textrm{PO}_{2} + \textrm{BaSO}_{4} \downarrow.
\end{equation}

The mixture was cooled down to $\sim$ 5 $^{\circ}\textrm{C}$ and left for 48 h to promote the Ostwald ripening of the small BaSO$_{4}$ crystals.
Then the solution was filtered using a vacuum pump and the filtrate was used for the next step when stoichiometric amount of hydrazine hydrate
was added to it under constant stirring.

\begin{equation}
\textrm{N}_{2}\textrm{H}_{4} + 2\textrm{H}_{3}\textrm{PO}_{2} \rightarrow \textrm{N}_{2}\textrm{H}_{6}(\textrm{H}_{2}\textrm{PO}_{2})_{2}.
\end{equation}

The obtained hydrazinium hypophosphite salt is highly soluble in water and thus no visible change in the solution
appearance was observed. The solution was evaporated slowly under vacuum ($\sim 400$ mbar) at temperature of 55 $\pm$ 5 $^{\circ}\textrm{C}$.


\end{document}